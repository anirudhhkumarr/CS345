\documentclass[a4paper,10pt]{article}
\usepackage{algpseudocode}
\usepackage{algorithm}
%opening

\begin{document}
\begin{center}{\Huge CS 345: Algorithms II}\\
\begin{flushright}
\textbf{Submitted By:}\\
Anirudh Kumar (Y9088)\\
Chandra Prakash (Y9181)
\end{flushright}
\LARGE \textbf{Assignment 3}\\
\end{center}
\normalsize
\section{Solution-1}
\begin{tabbing}
====\====\====\====\====\kill
assignEdgeWeight(S)\\
\{\\
\>$v\leftarrow S$\>\\
\>$visited[v]\leftarrow true$\\
\>$numPath[v]\leftarrow 0$\\
\>for each vertex $x$ such that $(v,x)\in E$\\
\>\>if($visited[x]==false)$\\
\>\>\>$W[v,x] = numPath[v]$\\
\>\>\>if($outDeg[x]==0$)\\
\>\>\>\>$numPath[x] \leftarrow 1$\\
\>\>\>else\\
\>\>\>\>assignEdgeWeight($x$)\\
\>\>\>$numPath[v] \leftarrow numPath[v]+numPath[x]$\\
\>\>else\\
\>\>\>$W[v,x] = numPath[v]$\\
\>\>\>$numPath[v] \leftarrow numPath[v]+numPath[x]$\\
\}\\
\end{tabbing}
\textbf{Proof of Algorithm}:\\
Induction:\\
\begin{description}
\item{Base Case:} Only one edge is present i.e $(u,v)$\\
$W[u,v] = 0$\\
$numPath[v] = 1$\\
$numPath[u] = 0 + numPath[v] = 1$\\
\item{Induction Step}\\
Let there be a graph G with $inDeg[u]=0$, $numPath[u]=m$, $outDeg[v]=0$ and all pathIDs are distinct and less than $m$.
Another node $w$ is added to G such that $inDeg[w]=0$ and has $n$ outgoing edges to the vertices of graph G namely $\{u_1,u_2,..u_i...u_n\}$.\\
Number of paths from $w$ to $v$ is given by \\\begin{center}$numPath[w] = \displaystyle\sum\limits_{i=1}^{n} numPath[u_i]$\end{center}
Now, consider an edge $(w,u_i)$ and the path $u_i$ to $v$. Let $W[w,u_i] = x$ then pathID of this path is $x + $pathID$(u_i,v)$. Then, by our algorithm $W[w,u_i] = x=\displaystyle\sum\limits_{j=1}^{i-1} numPath[u_j]$. Therefore, $x<numPath[w]$ and pathID$(u_i,v)$ is already less than $numPath[u_i]$(by induction).Therefore, pathID of each path less than the number of paths.\\
Now, we need to show that the new pathIDs are distinct.Let there be two paths $(w,u_i,v)$ and $(w,u_j,v)$ with same pathID then,\\
\begin{center}
$W[w,u_i]$ + pathID$(u_i,v)$ = $W[w,u_j]$ + pathID$(u_j,v)$\\
\end{center}
Without the loss of generality let $u_j$ be reached before $u_i$ in the loop. Then, \\$W[w,u_i]= W[w,u_j] + \displaystyle\sum\limits_{k=j}^{i-1} numPath[u_k]$.Therefore $W[w,u_i]\geq W[w,u_j]+numPath[u_j]$.\\We also have pathID$(u_j,v)<numPath[u_j]$(by induction).Therefore \\$W[w,u_i]$ + pathID$(u_i,v)$ > $W[w,u_j]$ + pathID$(u_j,v)$\\
\end{description}
Thus we have assigned unique pathIDs from $0\rightarrow N-1$ with each edge having integral edge weight.
\section{Solution-2}
\begin{algorithm}
 \caption{Optimising projects}
 \begin{algorithmic}
  \Function{Optimal}{$N, H$}
    \For{$i \leftarrow$ 0 to $H$}
     \State Optimal[$0$][$i$] $\gets$ 0
    \EndFor
    \For{$i \leftarrow$ 0 to $N$}
     \State Optimal[$i$][$0$] $\gets$ 0
    \EndFor
    \State $id \gets$ 1
    \For{$n \leftarrow$ 1 to $N$}
     \For{$h \leftarrow$ 1 to $H$}
      \State Optimal[$n$][$h$] $\gets$ max$_{i=0}^{h}$($f_{id}[i]$ + Optimal[$n-1$][$h-i$])
      \State Let the maximum be achieved for $i$
      \State Schedule Project $id$ for time i
      \State $id \gets id + 1$
     \EndFor
    \EndFor
  \EndFunction
 \end{algorithmic}
\end{algorithm}
\subsection{Explanation}
The above dynamic programming algorithm maintains an $O(NH)$ array Optimal such that Optimal[$n$][$h$]
represents the maximum total grade points for $n$ projects and $h$ hours to work on these projects. The
 optimal scheduling for $n$ projects and $h$ hours can be represented by the following recurrence:
\begin{center}
 Optimal[$n$][$h$] = max$_{i=0}^{h}$($f_{id}[i]$ + Optimal[$n-1$][$h-i$])
\end{center}
This recurrence essentially states that if we schedule a project for $i$ hours and 
then recursively calculate the optimal scheduling for the remaining projects in time $h-i$, and
maximise the total grade points over all $i$, we will get the optimal scheduling strategy for $n$ projects.
This is exactly what the above algorithm does. It initiates the first row and column with 0 
(0 projects $\rightarrow$ 0 grade points, 0 hours $\rightarrow$ 0 grade points). For each $n$, $h$ the values
required are Optimal[$n-1$][$0$] to Optimal[$n-1$][$0$] so they are already computed.
\subsection{Time Complexity}
The algorithm requires O($NH$) space for storing the matrix. The time complexity is O($NH^2$), as for calculting
the maximum we require O($H$) time and this is done for each entry in matrix.
\end{document}
