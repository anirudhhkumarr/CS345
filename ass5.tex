\documentclass[a4paper,12pt]{article}
\usepackage{algpseudocode}
\usepackage{algorithm}
\begin{document}

\begin{center}{\Huge CS 345: Algorithms II}\\
\begin{flushright}
\textbf{Submitted By:}\\
Anirudh Kumar (Y9088)\\
Chandra Prakash (Y9181)
\end{flushright}
\LARGE \textbf{Assignment 5}\\
\end{center}

\section{Solution of 1}
\subsection{Inference:}
The edge $(v,u)$ must be present in the shortest path that has been considered in the $i^{th}$ iteration.\\
If $(v,u)$ be a forward edge then for a non zero flow along this path, if $(u,v)$ is not present in the graph then this is introduce with capacity $f(v,u)$ in the residual graph. If $(v,u)$ is present as a backward edge the $(u,v)$ is introduced with residual capacity $c(u,v)-f(u,v)$. If $(v,u)$ is not in the path then $(u,v)$  can't be introduced by Ford-Fulkerson algorithm.
\subsection{Proof:}
Considering the most general case, before the $i^{th}$ there is a path $s\rightarrow a\rightarrow b\rightarrow t$ such that this is the shortest path, edge $(b,a)$ is not present in the graph and there is a path $a\rightarrow v$.Then in the $i^{th}$ iteration edge $(b,a)$ is introduced as result of which there is path $p_i(v)$=$s\rightarrow b\rightarrow a\rightarrow v$ then the claim is, $d_i(v)<d_{i-1}(v)$. We also have $d_{i-1}(a)<d_{i-1}(b)$ since the shortest path has been selected.Therefore
\begin{center}
$d_{i-1}(v)>d_i(v)>d_{i-1}(b)>d_{i-1}(a)$
\end{center}
Since $d_{i-1}(v) > d_{i-1}(a)$ therefore $d_{i}(a)\geq d_{i-1}(a)$ must satisfy.\\
But in this case $d_{i-1}(v) > d_{i-1}(v)$ $\Rightarrow $ path length of the path $s\rightarrow b\rightarrow a$ is less than the path length of the path $s\rightarrow a$(at the begining of the $i^{th}$ iteration). Therefore, $d_i(a)<d_{i-1}(a)$. Therefore contradiction.
\subsection{Proof:}
Using (2) our claim is, when $(u,v)$ is reappears in the graph, there must be an edge $(v,u)$ in the graph at the begining of the $j^{th}$ iteration. Therefore, $d_{j-1}(u)=d_{j-1}(v)+1$. Therefore, after the $j^{th}$ iteration, using (3) our claim is $d_j(u)>d_{j-1}(v)\geq d_{i}(v) = d_i(u)+1$. Therefore, $d_j(u)>d_i(u)$.
\subsection{Proof:}
During each iteration of FF algorithm, at least one edge disappears from the residual graph.Let $(u,v)$ be an edge that is removed in the $i^{th}$ iteration then, $(u,v)$ can appear atmost n times. If this is not the case then using (4) there would be a shorted path $d_k(u)>n$ ie $d_k(u)=\infty $ since the total number of vertices is n. Therefore, edge $(u,v)$ can't be reached. We have a total of m edges in the graph and each edge reappears atmost n times. Therefore, after $mn$ iterations there is no path between s and t. Therefore the algorithm terminates at this point and hence the running complexity is $O(m^2n)$.

\section{Solution of 2}
\subsection{Pseudo Code:}
\begin{algorithmic}
\Function{FindReachable}{$G$, $s$, $t$, $k$}
\While {there is an $s$-$t$ path in the graph G}
 \State Find a simple $s$-$t$ path in the graph G (no vertex repeated).
 \State Ping vertices on the path in order of a binary search to find the 
 \State nearest node on this path which is unreachable from $s$. Let this node
 \State be $v$ and the node immediately preceding it be $u$.
 \State Remove ($u$, $v$) from the graph G.
\EndWhile
\State Do a DFS of the remaining graph G from $s$ to find all the vertices
\State reachable from $s$.
\EndFunction
\end{algorithmic}
\subsection{Proof:}
The length of $s$-$t$ path can be atmost $n$ (no vertex is repeated). So if we do a binary search on 
this sequence of vertices we require atmost log($n$) ping operations. When we remove an edge ($u$, $v$) from the graph,
it is the one removed by the hacker too, because if there is an edge ($u$, $v$), there is no way to make $v$ unreachable 
and keep $u$ reachable without removing ($u$, $v$). Also since hacker has removed edges from min-size cut, ($u$, $v$) belongs
to min-size cut.\\
The algorithm terminates when there is no $s$-$t$ path remaining. Since we remove the edges from the min-size cut,
after $k$ iterations of while loop no edge from the min-size cut remains and consequently no $s$-$t$ path remains in the graph.
So the while loop runs $k$ times and in each iteration we execute a maximum of log($n$) ping operations.
So the algorithm executes O($k$log$n$) ping operations.\\
Finally we do a DFS on the remaining graph to find all the reachable vertices.
\subsection{Time Complexity}
The time complexity of ping($v$) ooperation has not been given so we represent it by P. The overall
time complexity of the algorithm can be represented as:
\begin{center}
 O(k(Plog(n) + n) + $|S|$)
\end{center}
This includes the time required to find an $s$-$t$ path in the graph using BFS i.e. O(n).
Second term is due to DFS done of the graph to report the set of reachable vertices. $|S|$ is the size
of the set returned by the algorithm. No node except those in $|S|$ are reachable from $s$, so DFS takes
O($|S|$) time.

\end{document}